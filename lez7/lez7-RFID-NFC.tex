\section{Radio Frequency Identification (RFID)}

	L'idea di RFID si è sviluppata come evoluzione elettronica dei codici a barre, per un'identificazione più rapida dei beni.
	
	La rete RFID si basa sulla presenza di due dispositivi di base: un lettore e un tag.
	Un lettore è composto da modulo a radiofrequenza (antenna), memoria, CPU, Batteria;
	un tag è composto da antenna, circuito di recupero dell'energia (trasmessa dal lettore), memoria non volatile per memorizzare l'ID, e in alcuni casi anche di sensori e una piccola CPU.
	Si distinguono 3 tipi di tag:
	\begin{description}
		\item[Passivi] Funzionano solo grazie al circuito di recupero dell'energia che cattura quella trasmessa dal lettore
		\item[Semi-passivi]  Dotati di batteria
		\item[Attivi] Dotati di batteria e trasmettitore
	\end{description}

	\paragraph{Electronic Product Code (EPC)}
	Ogni tag è dotato di un EPC, si tratta di uno standard che definisce un codice univoco del prodotto su 96 bit organizzati nel seguente modo:
	\begin{itemize}
		\item \textit{Header}. 8 bit, numero di versione del tag
		\item \textit{EPC Manager}. 28 bit, ID del produttore
		\item \textit{Object class}. 24 bit, ID del prodotto
		\item \textit{Serial number}. ID dell'unità
	\end{itemize}

	Le sfide nella realizzazione di un tag RFID sono la costruzione di un circuito di recupero dell'energia efficiente, la miniaturizzazione e il contenimento del costo.
	
	Le sfide nella realizzazione di un lettore RFID riguardano invece la realizzazione di antenne in grado di fornire sufficiente energia al tag, grande differenza nella magnitudo del segnale inviato e ricevuto e l'integrazione con i sistemi industriali più diffusi.
	
	\begin{figure}[h]
		\centering
		\includegraphics[width=0.8\textwidth]{lez7/useCasesRFID.png}
		\caption{Applicazioni reali che sfruttano RFID}
		\label{fig:useCasesRFID}
	\end{figure}

	RFID sfrutta le frequenze tra 100 e X MHz in quanto risultano essere quelle ottimali per la propagazione del segnale in ambienti affollati (Figura \ref{fig:frequenzaOttimale}).
	%completare valore X da appunti
	
	\begin{figure}[h]
		\centering
		\includegraphics[width=0.6\textwidth]{lez7/optimalFrequency.png}
		\caption{Performance di trasmissione al variare della frequenza per RFID}
		\label{fig:frequenzaOttimale}
	\end{figure}


\section{Near Field Communication (NFC)}

	\paragraph{Connessione tra due dispositivi}
	\begin{enumerate}
		\item La bobina nel primo dispositivo è attraversata da una corrente che genera un campo magnetico percepito dal secondo dispositivo
		\item il secondo dispositivo identifica il campo magnetico ricevuto come un segnale valido e offre la connessione
		\item il primo dispositivo accetta la connessione e inizia la transazione
	\end{enumerate}

	Grazie ai dispositivi dotati di NFC gli utenti possono
	eseguire pagamenti o usare coupon tramite il dispositivo, senza estrarre la carta di credito o debito;
	trasferire file;
	scaricare informazioni circa oggetti, servizi o luoghi;
	mostrare documenti digitale come le carte di imbarco.
	
	I rischi legati all'utilizzo di questa tecnologia sono essenzialmente legati
	alla privacy (quali dati vengono trasmessi, processati e memorizzati?),
	alla sicurezza (cosa succede se smarrisco lo smarthphone?)
	e al \emph{sentinel hacking}, in cui un tag NFC potrebbe essere nascosto in un punto e registrare le informazioni degli smartphone con cui viene in contatto.
	
	I principali vantaggi della tecnologia NFC restano comunque la comunicazione bi-direzionale, l'alto livello di sicurezza grazie alle tecniche di cifratura, l'alta velocità di riconoscimento con una minima probabilità di errore.
	
	\paragraph{Alternative}
	Un'alternativa a NFC sono i codici a barre o i codici QR.
	Ognuna di queste tecnologie presenta comunque alcuni punti di forza, motivo per cui NFC non mira a sostituire i lettori ottici.
	In Figura \ref{fig:comparazioneRFID-barcode} è riportato un confronto.
	
	\begin{figure}[h]
		\centering
		\includegraphics[width=0.8\textwidth]{lez7/tabellaComparativaRFID-barcode.png}
		\caption{Comparazione tra RFID e codici a barre}
		\label{fig:comparazioneRFID-barcode}
	\end{figure}
