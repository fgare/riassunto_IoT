\section{Livello di rete}

	Per una serie di ragioni legate all'elevato numero di nodi, alla limitata potenza e all'approccio data centrico non è possibile utilizzare un identificativo unico per ogni dispositivo all'interno di una sensor network.
	Un indirizzamento dei nodi è tuttavia necessario per la gestione dei nodi, il \emph{querying}, il \emph{service discovery} e il routing.
	
	Risulta necessario utilizzare nuovi algoritmi di routing che tengano conto della limitata potenza, capacità computazionale e memoria disponibili, dei frequenti cambiamenti nella topologia e dell'assenza di un identificativo globale dei dispositivi.
	
\paragraph{Tassonomia dei protocolli di routing}
	Si distinguono tre categorie di protocolli di routing
	\begin{itemize}
		\item Data Centric Protocols\\
			\underline{Flooding}, \underline{Gossiping}, \underline{SPIN}, SAR, \underline{Directed Diffusion}, Rumor Routing, Constrained Anisotropic Diffused Routing, COUGAR, ACQUIRE
		
		\item Hierarchical Protocols\\
			\underline{LEACH}, TEEN, APTEEN, PEGASIS, Energy Aware Scheme
		
		\item Location Based Protocols\\
			MECN, SMECN, GAF, GEAR
	\end{itemize}


\subsection{Flooding e Gossiping}

	Il \emph{flooding} è l'approccio convenzionale e consiste nell'invio broadcast dei dati a tutti i nodi vicini;
	il \emph{gossiping} prevede invece l'invio dei dati a un solo nodo vicino scelto casualmente.
	Sebbene semplici e reattive queste tecniche implicano una serie di svantaggi:
	\begin{itemize}
		\item Implosion
		\item Overlap
		\item Resource Blindness
		\item Power inefficient
	\end{itemize}

	Il gossiping è migliore rispetto al flooding in quanto invia i dati a un solo nodo per risparmiare energia ed evita implosion.
	
\paragraph{Protocollo di routing ideale}
	Le caratteristiche che il protocollo idea dovrebbe avere sono:
	\begin{itemize}
		\item selezione del percorso più breve per l'invio dei dati
		\item evitare overlap
		\item minimo consumo di energia
		\item conoscenza globale della topologia
	\end{itemize}


\subsection{SPIN: Sensor Protocol for Information via Negotiation}

	Alla base del protocollo SPIN ci sono due idee: i sensori si scambiano informazioni su dati che sono già in loro possesso o che desiderano avere, risparmiano così energia e lavorano in modo efficiente; i sensori devono monitorare ed adattarsi alle proprie risorse energetiche.
	
	Usa tre tipi di messaggi: ADV, REQ, DATA nel seguente modo:
	\begin{itemize}
		\item quando un sensore ha qualcosa da trasmettere invia in broadcast un \emph{advertisement packet} (ADV)
		\item i nodi interessati invio un \emph{request packet} (REQ)
		\item i dati sono inviati ai nodi che li richiedono
		\item la procedura viene ripetuta finché tutti i nodi non hanno una copia
	\end{itemize}

	SPIN è basato su \emph{data centric routing}, ossia i nodi inviano in broadcast l'advertisement nel caso ci siano dati disponibili e aspettano le richieste delle sink interessate, risulta quindi un ottimo protocollo per disseminare le informazioni tra tutti i nodi.
	
\paragraph{SPIN-1}
	Si tratta di un protocollo di handshake a tre fasi, ha il vantaggio di essere semplice e di evitare l'implosione ma al contempo implica un elevato consumo di potenza a causa dell'overhead introdotto.
	
\paragraph{SPIN-2}
	Si tratta di una variante di SPIN-1 che modifica il comportamento del sensore sulla base dell'energia disponibile.
	Quando la disponibilità di energia è massima il protocollo si comporta come SPIN-1, quando invece l'energia residua scende al di sotto di una certa soglia il nodo riduce la propria partecipazione, attivandosi solo se dispone di un'autonomia sufficiente a fargli completare le fasi rimanenti (stato di \emph{dormant}).
	
\paragraph{Conclusioni}
	\begin{description}
		\item[SPIN-1] Previene l'invio di messaggi ridondanti e permette un risparmio di energia del 70\% rispetto al flooding
		\item[SPIN-2] risparmio energetico del 10\% rispetto a SPIN-1 e del 60\% rispetto a flooding
	\end{description}
	
	
\subsection{Directed Diffusion Routing Algorithm}

	Si tratta di un algoritmo \emph{data-centric} in quanto un sensore non necessita di un'identità univoca ma ogni dato trasmesso è identificato mediante i suoi attributi.
	
\paragraph{Naming scheme}
	I dati generati dai sensori sono identificati dalla coppia attributo-valore; per generare una query occorre usare una coppia attributo-valore come nome dell'oggetto.
	
\paragraph{Directed Diffusion vs SPIN} \mbox{}\\[1ex]
	\begin{tabularx}{\textwidth}{XX}
		\toprule
		Directed Diffusion & SPIN\\
		\midrule
		Sink interroga i sensori per verificare la disponibilità di un dato inviando tramite flooding una regola. & I sensori dichiarano la disponibilità di dati, lasciando alla sink la possibilità di richiederli.\\
		\bottomrule
	\end{tabularx}

\paragraph{Vantaggi e svantaggi} \mbox{}\\[1ex]
	\begin{tabularx}{\textwidth}{XX}
		\toprule
		Vantaggi & Svantaggi\\
		\midrule
		\begin{itemize}
			\item DD è data centric, non necessita quindi di uno schema di indirizzamento
			\item ogni nodo può fare aggregazione, caching e sensing
			\item DD è efficiente a livello energetico in quanto è \emph{on demand} e non necessita di conoscere la topologia globale della rete
		\end{itemize}
		& \begin{itemize}
			\item Non è sempre utilizzabile in quanto si tratta di un \emph{query driven data delivery model}
			\item non è una buona scelta per applicazioni che necessita di un continuo invio di dati
			\item i \emph{naming schemes} sono dipendenti dall'applicazione
			\item il processo di abbinamento tra dato e query causa un leggero overhead
		\end{itemize}\\
		\bottomrule
	\end{tabularx}


\subsection{Low Energy Adaptive Clustering Hierarchy (LEACH)}
	
	LEACH è un protocollo basato su cluster con il fine di minimizzare la dissipazione di energia nelle reti di sensori.
	L'idea alla base è quella di selezionare casualmente alcuni sensori come cluster head e generare i cluster sulla base dell'intensità del segnale ricevuto.\\
	Si distinguono due fasi: \emph{set-up phase} e \emph{steady phase}.
	
\paragraph{Fase di setup}
	Nella fase di set-up ogni sensore genera un numero casuale tra 0 e 1, se il numero è minore di una certa soglia il nodo diventa cluster head;
	dopo che i cluster head sono stati selezionati i cluster head stessi comunicano agli altri nodi che sono i nuovi cluster head;
	terminata la fase di set-up ogni nodo accede alla rete attraverso il cluster head che richiede la minore energia per essere raggiunto.\\
	Quando i nodi ricevono l'advertisement determinano il cluster di appartenenza sulla base dell'intensità del segnale ricevuto, informano quindi il cluster head di voler partecipare al cluster; a questo punto i cluster head assegnano ai nodi l'intervallo di tempo in cui possono trasmettere.
	
\paragraph{Fase steady}
	I sensori rilevano e trasmettono dati ai cluster head, i quali aggregano i dati. Dopo un certo periodo trascorso nella fase steady, la rete torna nella fase di setup e ricomincia con la selezione di nuovi cluster head.
	
	E' difficile determinare il numero ottimale di cluster: pochi cluster significa che i nodi si trovano lontani dal proprio cluster head, al contrario molti cluster implica che molti nodi inviano dati alla sink.
	
\paragraph{Vantaggi}
	LEACH presenta numerosi vantaggi: 
	\begin{itemize}
		\item risparmio energetico rispetto a comunicazioni dirette
		\item clustering dinamico che permette di sopperire agevolmente alla perdita di alcuni nodi
		\item l'essere completamente distribuito e non richiedere la conoscenza globale della rete
		\item adozione di un single hop routing, ossia ogni nodo trasmette direttamente al cluster head
	\end{itemize}
	Tuttavia presenta lo svantaggio di non essere applicabile per regioni molto ampie.
		
	
	
	
