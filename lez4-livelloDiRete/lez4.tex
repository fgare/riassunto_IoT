\section{Livello di rete}

	Per una serie di ragioni legate all'elevato numero di nodi, alla limitata potenza e all'approccio data centrico non è possibile utilizzare un identificativo unico per ogni dispositivo all'interno di una sensor network.
	Un indirizzamento dei nodi è tuttavia necessario per la gestione dei nodi, il \emph{querying}, il \emph{service discovery} e il routing.
	
	Risulta necessario utilizzare nuovi algoritmi di routing che tengano conto della limitata potenza, capacità computazionale e memoria disponibili, dei frequenti cambiamenti nella topologia e dell'assenza di un identificativo globale dei dispositivi.
	
\paragraph{Tassonomia dei protocolli di routing}

	Si distinguono tre categorie di protocolli di routing
	\begin{itemize}
		\item Data Centric Protocols\\
			\underline{Flooding}, \underline{Gossiping}, \underline{SPIN}, SAR, \underline{Directed Diffusion}, Rumor Routing, Constrained Anisotropic Diffused Routing, COUGAR, ACQUIRE
		
		\item Hierarchical Protocols\\
			\underline{LEACH}, TEEN, APTEEN, PEGASIS, Energy Aware Scheme
		
		\item Location Based Protocols\\
			MECN, SMECN, GAF, GEAR
	\end{itemize}

\subsection{Flooding e Gossiping}

	Il \emph{flooding} è l'approccio convenzionale e consiste nell'invio broadcast dei dati a tutti i nodi vicini;
	il \emph{gossiping} prevede invece l'invio dei dati a un solo nodo vicino scelto casualmente.
	Sebbene semplici e reattive queste tecniche implicano una serie di svantaggi:
	\begin{itemize}
		\item Implosion
		\item Overlap
		\item Resource Blindness
		\item Power inefficient
	\end{itemize}

	Il gossiping è migliore rispetto al flooding in quanto invia i dati a un solo nodo per risparmiare energia ed evita implosion.
	
\paragraph{Protocollo di routing ideale}
	
	Le caratteristiche che il protocollo idea dovrebbe avere sono:
	\begin{itemize}
		\item selezione del percorso più breve per l'invio dei dati
		\item evitare overlap
		\item minimo consumo di energia
		\item conoscenza globale della topologia
	\end{itemize}

\subsection{SPIN: Sensor Protocol for Information via Negotiation}

	Alla base del protocollo SPIN ci sono due idee: i sensori si scambiano informazioni su dati che sono già in loro possesso o che desiderano avere, risparmiano così energia e lavorano in modo efficiente; i sensori devono monitorare ed adattarsi alle proprie risorse energetiche.
	
	Usa tre tipi di messaggi: ADV, REQ, DATA nel seguente modo:
	\begin{itemize}
		\item quando un sensore ha qualcosa da trasmettere invia in broadcast un \emph{advertisement packet} (ADV)
		\item i nodi interessati invio un \emph{request packet} (REQ)
		\item i dati sono inviati ai nodi che li richiedono
		\item la procedura viene ripetuta finché tutti i nodi non hanno una copia
	\end{itemize}

	SPIN è basato su \emph{data centric routing}, ossia i nodi inviano in broadcast l'advertisement nel caso ci siano dati disponibili e aspettano le richieste delle sink interessate, risulta quindi un ottimo protocollo per disseminare le informazioni tra tutti i nodi.
	
	
	
	
	
	
	
	
