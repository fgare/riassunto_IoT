\section{Livello di rete}

	Per una serie di ragioni legate all'elevato numero di nodi, alla limitata potenza e all'approccio data centrico non è possibile utilizzare un identificativo unico per ogni dispositivo all'interno di una sensor network.
	Un indirizzamento dei nodi è tuttavia necessario per la gestione dei nodi, il \emph{querying}, il \emph{service discovery} e il routing.
	
	Risulta necessario utilizzare nuovi algoritmi di routing che tengano conto della limitata potenza, capacità computazionale e memoria disponibili, dei frequenti cambiamenti nella topologia e dell'assenza di un identificativo globale dei dispositivi.
	
	\paragraph{Tassonomia dei protocolli di routing}
	Si distinguono tre categorie di protocolli di routing
	\begin{itemize}
		\item Data Centric Protocols\\
			\underline{Flooding}, \underline{Gossiping}, \underline{SPIN}, SAR, \underline{Directed Diffusion}, Rumor Routing, Constrained Anisotropic Diffused Routing, COUGAR, ACQUIRE
		
		\item Hierarchical Protocols\\
			\underline{LEACH}, TEEN, APTEEN, PEGASIS, Energy Aware Scheme
		
		\item Location Based Protocols\\
			MECN, SMECN, GAF, GEAR
	\end{itemize}

	\begin{itemize}
		\item Flooding
		\item Gossiping
		\item SPIN-1
		\item SPIN-2
		\item Directed Diffusion
		\item LEACH
	\end{itemize}