\section{Nanotecnologie}

	La nanotecnologia è un ramo della scienza applicata e della tecnologia che si occupa del controllo della materia su scala dimensionale nell'ordine del nanometro e della progettazione e realizzazione di dispositivi in tale scala.
	
	La nascita di questa nuova branca della tecnologia ci pone subito di fronte ad alcuni rischi, quali effetti avversi sulla salute umana o sull'ambiente in seguito ad un'esposizione volontaria o accidentale e la potenziale proprietà esplosiva delle nanostrutture.
	Risulta molto difficile fare una valutazione dei rischi associati a questa tecnologia per diversi fattori,
	in primo luogo c'è bisogno di personale specializzato e apparecchiature sofisticate;
	risulta poi difficile prevedere come determinate particelle si comporteranno una volta ingerite o disperse nell'ambiente; infine bisogna valutare la potenziale presenza di sostanze tossiche e la loro persistenza all'interno del corso o in natura.
	
	Al momento sono attive due grosse aree di ricerca: \emph{Electromagnetic Nano Communication} e \emph{comunicazione molecolare}.
	Nella comunicazione molecolare (\emph{molecular communication}) l'informazione è codificata all'interno di DNA, proteine, peptidi, etc ed è trasmessa per diffusione o trasporto attivo\footnote{Il trasporto attivo è il trasporto di molecole attraverso la membrana plasmatica mediato da una proteina transmembrana detta trasportatore di membrana.
		A differenza di quanto avviene nel trasporto passivo, nel trasporto attivo è richiesta una spesa energetica ed è sempre necessaria la mediazione di un trasportatore.
		In questa forma di trasporto le molecole si muovono contro un gradiente elettrico, chimico o elettrochimico.}.
	
	Nella comunicazione elettromagnetica si viene invece a creare una BAN (Body Area Network), composta da molti sensori che comunicano con un micro-gateway (tramite frequenze nell'ordine del THz).