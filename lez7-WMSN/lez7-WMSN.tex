\section{Wireless Multimedia Sensors Networks (WMSN)}

	La disponibilità di videocamere e microfoni miniaturizzati ha permesso la nascita delle reti di sensori multimediali, in cui vengono trasmessi segnali audio-video dell'ambiente che si desidera monitorare.
	L'alto volume di flussi multimediali insieme con la ricchezza di informazione generano problemi di congestione, privacy e sicurezza all'interno della rete.
	La prima sfida da affrontare è quella di limitare gli effetti della perdita di pacchetti, un pacchetto perso infatti non solo peggiora la qualità del segnale audio-video ricostruito, ma aumenta anche il consumo energetico a causa della ritrasmissione.
	
	I parametri da monitorare per evitare la congestione della rete sono: carico del canale, tempo di arrivo tra i pacchetti, occupazione del buffer locale.
	
	
\subsection{Secure Selective Dropping Congestion Control (S\textsuperscript{2}DCC)}

	Il protocollo S2DCC
