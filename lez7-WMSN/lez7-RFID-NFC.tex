\section{Radio Frequency Identification (RFID)}

	L'idea di RFID si è sviluppata come evoluzione elettronica dei codici a barre, per un'identificazione più rapida dei beni.
	
	La rete RFID si basa sulla presenza di due dispositivi di base: un lettore e un tag.
	Un lettore è composto da modulo a radiofrequenza (antenna), memoria, CPU, Batteria;
	un tag è composto da antenna, circuito di recupero dell'energia (trasmessa dal lettore), memoria non volatile per memorizzare l'ID, e in alcuni casi anche di sensori e una piccola CPU.
	Si distinguono 3 tipi di tag:
	\begin{description}
		\item[Passivi] Funzionano solo grazie al circuito di recupero dell'energia che cattura quella trasmessa dal lettore
		\item[Semi-passivi]  Dotati di batteria
		\item[Attivi] Dotati di batteria e trasmettitore
	\end{description}

	\paragraph{Electronic Product Code (EPC)}
	Si tratta di uno standard che definisce un codice univoco del prodotto a 96 bit composto nel seguente modo:
	\begin{itemize}
		\item \textit{Header}. 8 bit, numero di versione del tag
		\item \textit{EPC Manager}. 28 bit, ID del produttore
		\item \textit{Object class}. 24 bit, ID del prodotto
		\item \textit{Serial number}. ID dell'unità
	\end{itemize}